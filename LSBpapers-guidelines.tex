\documentclass[times,linguex]{lsb}
% possible options: 
% [times] for Times New Roman font. The font used is TeX Gyre Termes, which can be dowloaded free from https://www.fontsquirrel.com/fonts/tex-gyre-termes
% [cm] for Computer Modern font 
% [lucida] for Lucida font (not freely available)
% [brill] open type font, freely downloadable for non-commercial use from http://www.brill.com/about/brill-fonts; requires xetex
% [charis] for CharisSIL font, freely downloadable from http://software.sil.org/charis/
% [linguex] loads the linguex example package

\usepackage[linguistics]{forest} %for nice tree diagrams
\usepackage[british]{babel} %to get single quotes with command \enquote used in references list
\usepackage{multicol}
\usepackage{enotez} %for endnotes
\DeclareInstance{enotez-list}{custom}{paragraph}{
heading = \section*{#1} ,
notes-sep = 0\baselineskip ,
format = \footnotesize, %\leftskip0.6em ,
number = \textsuperscript{#1} ,
}
%\let\footnote=\endnote

\makeatletter
\renewcommand{\@seccntformat}[1]{\csname the#1\endcsname.\quad}
\makeatother

% \pdf* commands provide metadata for the PDF output. ASCII characters only!
\pdfauthor{}
\pdftitle{}
\pdfkeywords{}

% Optional short title inside square brackets, for the running headers. If no short title is given, no title appears in the headers.

\title[Author Guidelines]{Papers of the LSB\\ 
Studies van de BKL\\
Travaux du CBL\\
\Large Author guidelines}


\author[Guido Vanden Wyngaerd]% short form of the author name(s) for the running header
{
\spauthor{Guido Vanden Wyngaerd \\ 
\institute{KU Leuven}} 
%\AND 
%\spauthor{Author2 \\  \institute{Linguistic Institute}}
}
%as many authors as you like, each separated by \AND.

\begin{document}

\maketitle

\begin{abstract} \normalsize  This document provides a full overview of the information relating to submissions for the Papers of the LSB. This information includes (i) the stylesheet, and (ii) further author guidelines. So as to provide instruction both by example and by rule, this document has been formatted in accordance with the stylesheet it contains.
\end{abstract}

\clearpage

\section{General}

Manuscripts are accepted in Dutch, French, English and German. Should your contribution be in English, please make a consistent choice between British and American spelling. When submitting the initial manuscript to the editors of the volume, make your paper anonymous, and  add a separate sheet with the full title of the work, your full name, affiliation, as well as current mailing and e-mail address.

The contributions should not exceed the limit of 4600 words (including footnotes, references etc.). 

\section{Copyright}

The Papers of the LSB are Open Access articles distributed under the terms of the Creative Commons Attribution 4.0 license (CC BY 4.0: \url{http://creativecommons.org/licenses/by/4.0}), which permit unrestricted use, distribution, and reproduction in any medium, provided the original work is properly cited. Copyright is retained by the author(s).

\section{Files}

Please name the file containing your submission with the first three letters of the (first) author’s last name. Do not use the three/four-character extension for things other than the identification of the file type (smi.docx, not smi.new or smi.rev). 

\section{Templates}

\subsection{\LaTeX\ }
Contributions in \LaTeX\ are strongly preferred. The necessary files are available at \url{https://github.com/guidovw/LSBlatex}. The current document, which is available as a tex file, can be used as a template by replacing the document text with the text of your paper.

\subsection{MS Word}

A template for MS Word files (BKLtempl.dot) and an example document are also available at the above web address. Click on the files, and then click on the Download button. Make use of the electronic template, as this automatically takes care of the most important formatting aspects of your document. 

Please also use the appropriate predefined styles in the template.

\ex. \begin{tabular}[t]{llll@{\hskip 36pt}}
Title			& Title of your article\\
Subtitle		& Subtitle of your article\\
Authors		& Author name(s)\\
Affiliation	& Your affiliation\\
Abstract	& 	Abstract\\
utitle0,u	& 	Unnumbered section heading (for notes, references)\\
Vtitle1, v	& 	Section heading, first level\\
Wtitle2, w	& 	Section heading, second level\\
Xtitle 3, x	& 	Section heading, third level\\
Normal,s & Running text \\
Quote, q	& 	Block quotation\\
Table, t	& 	Table\\
Figure, f	& 	Figure\\
Example, e	& 	Example\\
Notes, n	& 	Notes\\
References, r	& 	References\\
Scaps, sc	& 	small caps\\
IPA		& 	IPA signs \\
/1, /2, /4	& 	White line (whole, half, quarter)   \\
\end{tabular}

The running heads will be generated automatically if you have filled in the appropriate information under File – Properties – Summary info.


\subsection{Margins}

It is not possible to fix the settings of the margins in the electronic style, and copying material into the document may change the margin settings. Please check the margin settings before you start formatting your article and before you submit it to the editors. The margin settings should be as in \Next  (on the basis of A4 paper size):

\ex. \begin{tabular}[t]{llll@{\hskip 36pt}}
Top 	&	7,20 cm\\
Bottom &	5,85 cm\\
Left 	&	4,75 cm\\
Right &		4,75 cm\\
Header &	6,20 cm\\
Footer 	&	1,25 cm\\
\end{tabular}


\section{Font enhancements}

Font enhancements (such as italics, bold, caps, small caps, etc.) within the text must be supplied by you. Whatever formatting or style conventions are employed, please be consistent.

Please use italics for foreign language words, highlighting and emphasis. Bold should be used only for highlighting within italics and for headings. Please refrain from the use of FULL CAPS (except for focal stress and abbreviations) and underlining (except for highlighting within examples, as an alternative for boldface), unless this is a strict convention in your field of research.


\section{Quotations}

In the main text quotations should be given in double quotation marks. Quotations longer than 3 lines should be formatted as block quotations, without quotation marks and with the appropriate reference to the source.


\section{Examples and glosses}

Examples should be numbered with Arabic numerals in parentheses.

\ex. \emph{Japanese}\\ %optional line for the name of the language (italics), source, etc. Note the absence of \exg., instead use \ex. (and \a. \b. for subdivisions) when this optional line is present. Use \exg. etc. if this line is absent
\gll Kare wa besutoseraa o takusan kaite-iru\\ %original foreign language example preceded by \gll
he \textsc{top} best-seller \textsc{acc} many write-\textsc{pfv}\\ %gloss line
\glt ‘He has written many best-sellers.’ %translation, preceded by \glt

\sloppy
Please note that the interlinear gloss gets no punctuation and no highlighting. In MS Word, the example and its gloss are lined up through the use of spaces; make sure the number of elements on both lines match. For the abbreviations in the interlinear gloss use small caps. For more information on glossing conventions, consult the Leipzig Glossing Rules (\url{https://www.eva.mpg.de/lingua/pdf/Glossing-Rules.pdf}).  If you use different abbreviations than the standard ones in the Leipzig Glossing Rules, add a list of abbreviations at the end of your paper (just before the References).

\section{Tables and figures}

Number your tables and figures consecutively, and provide  appropriate captions. The caption goes at the top of tables and at the bottom of figures.

In \LaTeX, tables and figures are treated as floats. This means that their placement on the page will not necessarily be where you put it in your manuscript, as this may lead to large parts of the page ending up white (e.g. when a table or figure does not fit on the current page anymore and wraps onto the following page). For this reason, you must always refer to tables and figures in the running text, as in the following example: ``In certain languages, the superlative transparently contains the comparative morphologically, as illustrated in Table \ref{tbl:table1} \citep[46]{Bobaljik2012}.'' Do not refer to tables and figures using the words `following', `below' or `above', as the final placement of your table or figure may be different from where you put them in your manuscript.

\begin{table}[h]
\centering
\caption{Morphological containment}	
\begin{tabular}{lllll}
 & \textsc{pos} & \textsc{cmpr} & \textsc{sprl}\\
\hline 
Persian & kam & kam-tar & kam-tar-in & ‘little’\\
Cimbrian & šüa & šüan-ar & šüan-ar-ste & ‘pretty’ \\
Czech & mlad-ý & mlad-ší & nej-mlad-ší & ‘young’\\
Hungarian & nagy & nagy-obb & leg-nagy-obb & ‘big’\\
Latvian & zil-ais & zil-âk-ais & vis-zil-âk-ais & ‘blue’\\
Ubykh &  nüs\textsuperscript{w}ə & ç’a-nüs\textsuperscript{w}ə & a-ç’a-nüs\textsuperscript{w}ə & ‘pretty’ \\
\end{tabular}\label{tbl:table1}
\end{table}


\section{Notes}

Notes should be kept to a minimum and not be used for references. Note indicators in the text should appear at the end of sentences and follow punctuation marks. Use footnotes rather than endnotes.

\section{References}

It is essential that the references are formatted according to these guidelines. Examples of references in the text:

\ex.[]
 ``Hayes (1995a) has proposed to use \ldots''\\
``\ldots\ if it is followed by a yer (Rubach 1993:144).''\\
``\ldots\ (see for example Yarmohammadi 1973, and Baynham 1991).''

For examples of references in the reference section: see the next section.

%\printendnotes[custom] 

\nocite{*} %this is to get all the entries of the sample bibliography; delete this line for an actual submission
\bibliography{sample} %change 'sample' by the name of your bib-file


\end{document}